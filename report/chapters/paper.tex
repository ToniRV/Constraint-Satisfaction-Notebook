\chapter{Maintaining Arc Consistency}

While forward checking (\cref{bt_with_forward_checking}) detects many inconsistencies, it does not detect all of them.
The problem is that forward checking only makes the current variable arc-consistent, yet it does not proceed further, checking that other variables are still themselves arc-consistent.
\TODO{Show example where forward-checking would not detect an inconsistency similar to the one presented in AIMA}

The algorithm called \textbf{Maintaining Arc Consistency (MAC)} \cite{10.1007/3-540-58601-6_86} uses as \texttt{Inference} function in \cref{backtracking} the AC-3 algorithm (see \cref{ac3}).
In particular, after a variable $X_i$ is assigned a value, MAC calls AC-3 with the arcs $(X_i, X_j)$ for all $X_j$ that are unassigned variables neighboring $X_i$ (instead of using AC-3 with all the arcs in the CSP).

From there, AC-3 does constraint propagation in the usual way, and if any variable has its domain reduced to the empty set , the call to AC-3 fails, thereby backtracking immediately.

MAC is strictly more powerful than forward checking because forward checking does the same thing as MAC on the initial arcs in MAC's queue; but unlike MAC, forward checking does not \textbf{recursively} propagate constraints when changes are made to the domains of variables.

\TODO{Since at this point we would have already gone through all functions in \cref{backtracking} and implemented them together with the reader, presenting MAC should not take too much time, as we would just pick the AC-3 algorithm that we implemented previously and plug it into the backtracking \cref{backtracking}.}

